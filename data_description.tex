\chapter{Descrizione dei dati}
\label{ch:data}

Di seguito vengono descritti i 4 dataset utilizzati nella sperimentazione

\section{German Credit dataset}
Il dataset contiene informazioni in ambito finanziaro su clienti ritenuti a rischio o meno.

\begin{itemize}
	\item Numero di istanze: 1000
	\item Numero di attributi: 21
	\item Attributo target: \textbf{class}
	\item Valori target: \texttt{\{good, bad\}}
\end{itemize}

\begin{table}[!htb]
	\centering
	\begin{tabular}{|r|l|}
		\hline
		Attributo & Tipo \\
		\hline
		checking\_status & nominal \\
		duration & numeric \\
		credit\_history & nominal \\
		purpose & nominal \\
		credit\_amount & numeric \\
		savings\_status & nominal \\
		employment & nominal \\
		installment\_commitment & numeric \\
		personal\_status & nominal \\
		other\_parties & nominal \\
		residence\_since & numeric \\
		property\_magnitude & nominal \\
		age & numeric \\
		other\_payment\_plans & nominal \\
		housing & nominal \\
		existing\_credits & numeric \\
		job & nominal \\
		num\_dependents & numeric \\
		own\_telephone & nominal \\
		foreign\_worker & nominal \\
		\textbf{class} & nominal \\
		\hline
	\end{tabular}
\end{table}

\pagebreak

\section{Hepatitis dataset}

Il dataset contiene informazioni su un vari casi di epatite.


\begin{itemize}
	\item Numero di istanze: 135
	\item Numero di attributi: 20
	\item Attributo target: \textbf{Class}
	\item Valori target: \texttt{\{DIE, LIVE\}}
\end{itemize}

\begin{table}[ht]
	\centering
	\begin{tabular}{|r|l|}
		\hline
		Attributo & Tipo \\
		\hline
		AGE & numeric \\
		SEX & nominal \\
		STEROID & nominal \\
		ANTIVIRALS & nominal \\
		FATIGUE & nominal \\
		MALAISE & nominal \\
		ANOREXIA & nominal \\
		LIVER\_BIG & nominal \\
		LIVER\_FIRM & nominal \\
		SPLEEN\_PALPABLE & nominal \\
		SPIDERS & nominal \\
		ASCITES & nominal \\
		VARICES & nominal \\
		BILIRUBIN & numeric \\
		ALK\_PHOSPHATE & numeric \\
		SGOT & numeric \\
		ALBUMIN & numeric \\
		PROTIME & numeric \\
		HISTOLOGY & nominal \\
		\textbf{Class} & nominal \\
		\hline
	\end{tabular}
\end{table}

\pagebreak

%\section{Image Segmentation dataset}
%
%Il dataset contiente informazioni di sette immagini all'aperto che sono state segmentate a mano. Ogni istanza rappresenta una regione 3x3.
%
%\begin{itemize}
%	\item Numero di istanze: 2310
%	\item Numero di attributi: 20
%	\item Attributo target: \textbf{class}
%	\item Valori target: \texttt{\{brickface, sky, foliage, cement, window, path, grass\}}
%\end{itemize}
%
%\begin{table}[!htb]
%	\centering
%	\begin{tabular}{|r|l|}
%		\hline
%		Attributo & Tipo \\
%		\hline
%		region-centroid-col & numeric \\
%		region-centroid-row & numeric \\
%		region-pixel-count & numeric \\
%		short-line-density-5 & numeric \\
%		short-line-density-2 & numeric \\
%		vedge-mean & numeric \\
%		vegde-sd & numeric \\
%		hedge-mean & numeric \\
%		hedge-sd & numeric \\
%		intensity-mean & numeric \\
%		rawred-mean & numeric \\
%		rawblue-mean & numeric \\
%		rawgreen-mean & numeric \\
%		exred-mean & numeric \\
%		exblue-mean & numeric \\
%		exgreen-mean & numeric \\
%		value-mean & numeric \\
%		saturation-mean & numeric \\
%		hue-mean & numeric \\
%		\textbf{class} & nominal \\
%		\hline
%	\end{tabular}
%\end{table}
%
%\pagebreak

\section{Vehicle Silhouettes dataset}

Il dataset contiene informazioni per discriminare le silhouette di diversi veicoli tra automobili, van e bus.

\begin{itemize}
	\item Numero di istanze: 846 
	\item Numero di attributi: 19
	\item Attributo target:  \textbf{Class}
	\item Valori target: \texttt{\{opel, saab, bus, van\}}
\end{itemize}


\begin{table}[!htb]
	\centering
	\begin{tabular}{|r|l|}
		\hline
		Attributo & Tipo. \\
		\hline
		COMPACTNESS & numeric \\
		CIRCULARITY & numeric \\
		DISTANCE CIRCULARITY & numeric \\
		RADIUS RATIO & numeric \\
		PR.AXIS ASPECT RATIO & numeric \\
		MAX.LENGTH ASPECT RATIO & numeric \\
		SCATTER RATIO & numeric \\
		ELONGATEDNESS & numeric \\
		PR.AXIS RECTANGULARITY & numeric \\
		MAX.LENGTH RECTANGULARITY & numeric \\
		SCALED VARIANCE\_MAJOR & numeric \\
		SCALED VARIANCE\_MINOR & numeric \\
		SCALED RADIUS OF GYRATION & numeric \\
		SKEWNESS ABOUT\_MAJOR & numeric \\
		SKEWNESS ABOUT\_MINOR & numeric \\
		KURTOSIS ABOUT\_MAJOR & numeric \\
		KURTOSIS ABOUT\_MINOR & numeric \\
		HOLLOWS RATIO & numeric \\
		\textbf{Class} & nominal \\
		\hline
	\end{tabular}
\end{table}

\pagebreak

\section{Wisconsin Breast Cancer dataset}

Il dataset contiene informazioni riguardo a vari casi di tumore al seno, che permettono di stabilire se esso è benigno o maligno.

\begin{itemize}
	\item Numero di istanze: 699
	\item Numero di attributi: 10
	\item Attributo target: \textbf{Class}
	\item Valori target: \texttt{\{benign, malignant\}}
\end{itemize}


\begin{table}[!htb]
	\centering
	\begin{tabular}{|r|l|}
		\hline
		Attributo & Tipo \\
		\hline
		Clump\_Thickness & numeric \\
		Cell\_Size\_Uniformity & numeric \\
		Cell\_Shape\_Uniformity & numeric \\
		Marginal\_Adhesion & numeric \\
		Single\_Epi\_Cell\_Size & numeric \\
		Bare\_Nuclei & numeric \\
		Bland\_Chromatin & numeric \\
		Normal\_Nucleoli & numeric \\
		Mitoses & numeric \\
		\textbf{Class} & nominal \\
		\hline
	\end{tabular}
\end{table}
